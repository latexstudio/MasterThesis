%%
%% 第四章
%% 2014.6.27
\chapter{重建算法系统设计及仿真结果}

\section{系统设计}

本文综合使用上述两章提到的图像重建算法和相似图像搜索技术,在文献\cite{Yue:2013gl}系统的基础上进行了完善,设计了针对大数据集的图像重建系统。系统分为离线训练和在线搜索两部分组成,如图\ref{fig:system}所示。

\begin{figure}
\centering\includegraphics[width=15cm]{imgs/ch4/system}
\caption{服务器在线重建流程图}
\label{fig:system}
\end{figure}

\subsection{离线训练系统}
服务器离线训练系统主要是针对大图像数据集构建索引,分为这样几个步骤:
\begin{enumerate}
\item 提取数据库中每一幅图像的局部特征,本文使用的是SIFT特征。SIFT特征需要进行关键点检测、二次曲线拟合关键点精确定位、消除边缘响应、梯度方向直方图计算等一些列操作,运算量大,相对耗时。因为每一幅图像是独立的,这一步可以使用并行算法,将图像分布式的存储在多台机器上,分别进行局部特征提取。
\item 对全体特征进行抽样,选择全体特征描述子的一个子集。
\item 对描述子子集使用上文提到的近似K均值聚类AKM,生成视觉词码表。视觉词码表将作为后续所有局部特征的量化标准,一旦对视觉词码表更新,所有局部特征对应的分类也需要更新。
\item 根据视觉词码表,对局部特征全集进行分类计算,计算最近邻的方式与AKM中寻找最近邻类目的计算方式一致,都是采用随机K-D树森林的方式进行高效查询。每一幅图像得到一个视觉词的集合。
\item 并行处理每一幅图像,根据每一个视觉词及其覆盖区域内的其它视觉词的尺度及空间位置关系进行编码,生成视觉词组。
\item 对视觉词组和视觉词构建倒排索引,生成倒排索引表。倒排索引表的结构如图\ref{fig:inverted}所示。
\end{enumerate}
\begin{figure}
\centering\includegraphics[width=13cm]{imgs/ch4/inverted}
\caption{相似图像查询用到的倒排索引结构}
\label{fig:inverted}
\end{figure}
其中倒排索引表由多层索引来表示以便后续的拓展。每个视觉词出现在多幅图像中,构成一对多关系,而每幅图像包含了多组视觉词组,每个视觉词组由多个视觉词以及该视觉词在当前词组中的编码信息构成。
进行相似搜索时,分别根据查询图像的每一个中心视觉词进行查询,根据中心视觉词的id找到它所对应的若干图像,再根据图像找到其相匹配的以相同id为中心视觉词的视觉词组,对两个匹配的视觉词组进行打分,累积相加作为当前候选图像的得分,将所有打过分的候选图形进行得分排序,即得到相似性由高到低的候选图像列表。

首次离线计算生成的视觉词码表一般无需改动,后续追加到系统中的图像文件按照上述4~6步骤进行索引,索引文件增量式更新。

\subsection{在线重建系统}
在线重建系统是针对客户端发送的请求,实时的进行图像重建。客户端-云端在线重建系统如图\ref{fig:serverOnline}所示。

\begin{figure}
\centering\includegraphics[width=15cm]{imgs/ch4/serverOnline}
\caption{服务器在线重建流程图}
\label{fig:serverOnline}
\end{figure}

从客户端传输来的压缩编码的数据包含了三部分信息,分别是(1)下采样图像,是1:256比例进行采样的图像,服务器端利用插值法进行上采样,恢复成原始大小得到\(I_u\);(2)SIFT关键点,传送给服务器的特征仅仅包含关键点的位置、尺度、方向等信息,不包含128维的描述子,服务器需要根据上述信息在图像\(I_u\)上重新计算特征描述,做进一步的匹配工作(3)控制信息,包括误差控制参数中的基准误差和图像块的尺度控制等,尺度控制数值与客户端提取SIFT特征计算时有关,在客户端提取时一幅图像包含各个尺度上大量的SIFT算子,通常我们需要设定一定的尺度门限,只传输门限以上的特征,这个门限可以通过SIFT算子的数量不同动态的设定。设定的门限对服务器端的候选图想块过滤起到一定的影响,所以我们将尺度门限作为控制参数传递给服务器,近一步指导候选图像块的筛选。

经过上述计算,我们解码得到三类数据,按照以下步骤进行图像重建:

\begin{enumerate}
\item 运用上文提到的视觉词组编码方式计算每一个图像块的视觉词组;
\item 对原始图像进行分块,分别使用每一块的视觉词组在大规模图像集上利用倒排索引进行快速的相似图像搜索,合并搜索结果,得到了候选图像集合;
\item 对候选图像进行SIFT提取后与原始图像的SIFT进行匹配得到候选图像块;
\item 使用误差控制阈值对候选图像块进行筛选,得到待融合图像块;
\item 在下采样图像的指导下使用经典的泊松图像编辑\cite{Perez:2003ul},通过卷积近似求解\cite{Farbman:2011dc},对图像块进行由大到小的无缝融合,最终得到一幅高分辨率的图像。
\end{enumerate}

\section{实验结果}
本文在Matlab和Python环境下仿真了上述的图像重建系统,使用VL-FEAT图像处理函数库
\cite{vl_feat}进行SIFT特征提取,使用其中相似K聚类(Approximately K-Means)进行视觉词的量化分类。数据集使用的是INRIA Holiday数据集\cite{INRIA},包含了500组共1491个人旅游照片,图像集中平均每幅图像的分辨率在五百万像素以上。选择了其中三十张图像进行重建,其它图像作为大图像数据集进行训练。以尺度大于3.2的所有SIFT算子作为局部特征的全集。

图\ref{fig:detail_result}显示了其中一幅图像的实验结果。

\subsection{图像重建效果及分析}
\begin{figure}
\centering\includegraphics[width=15cm]{imgs/ch4/detail_result}
\caption{图像重建结果,左边两幅分别是原始图像和还原图像,红色线框表示被放大显示的细节区域,右边两幅分别是局部放大的原始图像和还原图像}
\label{fig:detail_result}
\end{figure}

研究发现,重建图像出现局部模糊的原因主要出自三个方面:(1)原始图像特征图像块覆盖不完整,部分区域没有图像块进行覆盖;(2)候选图像集不完备,不能够涵盖请求图像的某个区域,比如图像中出现人脸等强区分性特征时,图像集会不包含相似的候选图像块的情况会有较大的出现几率;(3)候选图像集相对完备,但图像块筛选时将其过滤掉,阈值设置不够合理。当我们使用自适应阈值的图像块筛选法之后,这种情况出现的几率大大降低,请看下一节的实验结果。

\subsection{自适应阈值的图像块筛选实验结果}
本文针对自适应阈值算法做了对比实验,分别勘察使用自适应阈值前后图像的重建重建效果,此处列举其中三幅图像的实验结果,如图\ref{fig:result}所示。

图中第一列是原始图像,第二列是采用较大数值作为误差阈值时的重建结果,第三列是采用较小数值作为误差阈值时的重建结果,第四列是采用自适应阈值法时的重建结果。每幅图像有两行实验结果,第一行表示的是原始图像全部像素,红框标示出重建图像错误匹配的局部区域,第二行是错误区域的局部放大。

\begin{figure}
\centering\includegraphics[width=15cm]{imgs/ch4/rec_result}
\caption{自适应阈值实验结果}
\label{fig:result}
\end{figure}

可以看到第一幅图像在阈值较大、误差条件宽松时对图像的重建效果好,在条件严苛的时候,红色的车不能重建得到,如图中红框处所示;第二幅像在阈值较小、误差条件苛刻时还原效果好,在条件宽松时,会出现错误匹配,如雪山顶和天空衔接处红框所示;第三幅图像在误差阈值较大时出现错误匹配,而在较小时又有大面积的区域匹配不到任何候选图像块,如人群处红框所示。采用自动阈值的方式进行图像重建,三幅图像的重建的效果比较理想。实验结果表明,采用自动阈值进行候选图想块筛选的方式虽然不能保证每一幅图像都获得最佳的重建结果,但在大部分情况下它能保证还原的图像有着较佳的主观视觉效果。

\subsection{相似图像搜索实验结果}
我们对图像\ref{fig:candi_query}进行相似图像搜索,图\ref{fig:candidates1}和图\ref{fig:candidates2}分别显示了使用分区搜索前后的实验对比,每个搜索结果列出了前十二个候选图相似图像。
\begin{figure}
\centering\includegraphics[width=8cm]{imgs/ch4/candi_query}
\caption{进行相似图像搜索的请求图片}
\label{fig:candi_query}
\end{figure}

\begin{figure}
\centering\includegraphics[width=15cm]{imgs/ch4/candidates1}
\caption{不使用分区搜索的相似图像搜索}
\label{fig:candidates1}
\end{figure}

\begin{figure}
\centering\includegraphics[width=15cm]{imgs/ch4/candidates2}
\caption{使用分区搜索进行相似图像搜索}
\label{fig:candidates2}
\end{figure}
从实验结果我们看到,使用分区相似图像搜索得到的图片更具有多样性,特别是对大图像数据集进行相似图像搜索时,如果不进去分区搜索,搜索排名靠前的图像包含大量重复的图片,会有很大的概率导致原始图像的某些局部特征块不再候选图像集和中,使用分区搜索再合并的策略,这一问题得到有效的解决。

\ifx\usechapbib\empty
\nocite{BSTcontrol}
\bibliographystyle{buptgraduatethesis}
\bibliography{bare_thesis}
\fi
