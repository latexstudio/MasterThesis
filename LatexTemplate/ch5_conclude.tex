%%
%% This is file `example/ch_concln.tex',
%% generated with the docstrip utility.
%%
%% The original source files were:
%%
%% install/buptgraduatethesis.dtx  (with options: `ch-concln')
%% 
%% This file is a part of the example of BUPTGraduateThesis.
%% 

\chapter{总结与展望}

\section{全文总结}

本课题重点研究在基于云的图像压缩应用场景下的基于局部特征的图像重建算法,探索利用更丰富的图像特征信息在大语料集中进行高质量的图像重建,基于局部特征对图像进行重建的技术能够让我们在发送端只传输少量的特征数据,而接收端服务器利用大数据集进行高分辨率图像的还原。

\subsection{论文主要工作}

\begin{enumerate}
\item 本文在基于云的图像压缩传输这一应用场景下,对当前的基于云的图像重建算法的各个环节进行了深入的探讨,在其基础上针对大规模图像数据的使用进行了一定的完善,在相似图像搜索和图像重建等算法中提出了一定的改进方案;
\item 在计算机视觉领域综合运用图像关键点的提取、局部特征描述、特征匹配、图像配准、图像分割、图像融合等图像处理算法,形成一套完整的基于局部特征进行图像重建的算法流程;
\item 另针对大数据集的处理,由于图像的匹配与重建等相关任务是建立在大数据集基础之上的,所以本课题试图探索如何在大规模图像集上优化现有的图像重建方案,采用改进的聚类算法生成视觉词,利用k-d树等数据结构组织视觉词并进行快速的词条匹配;
\item 本文分别从基于相似图像集的图像重建和大规模近似重复图像搜索两个关键技术出发,构建了离线学习和在线重建相结合的基于局部特征的图像重建系统,完成了重建环节的所有流程,提出了优化措施并进行了大量实验;
\end{enumerate}

\subsection{论文成果}

\begin{enumerate}
\item 提出了自适应阈值的图像块筛选方法,实验结果表明:在不使用该方法时人工指定阈值在某些图像上能达到理想的还原效果,但在另外一些图像上则效果并不理想,结果数据显示由于不同图像之间不同区域的匹配块存在大量差异,没有一个固定的误差控制阈值能够适用于各类图像;而在重建系统中加入自适应阈值的方法,能够有效的找到合适的误差控制阈值,使得重建结果得到了提升,加强了图像重建系统的稳定性与适用性;
\item 提出了相似图像搜索的分区查询方案,实验结果显示分区后再进行相似图像搜索的方案显著提升了搜索结果组成的相关图像候选集的结果多样性,在大图像数据集上对图像重建效果有一定的提升;
\item 在视觉词袋模型基础上,提出了视觉词组编码方案进行粒度细分的搜索算法,提出了在尺度大小和空间位置两个维度上的编码方案,加强了视觉词组的鲁棒性,提升了相似搜索的精度。
\end{enumerate}

\subsection{论文问题分析}

重建结果在大部分实验数据上得到了较好的重建效果,但在个别样例上表现不佳,算法设计了离线训练和在线搜索两个环节,离线训练训练中的训练样本的选取、聚类个数的选取以及聚类起点的选取都会对相似搜索准确性的影响,如何对上述环节进行合理的参数选择以及如何在图像数据流不断累积的情况下进行增量式的学习有待进一步的探索;

\section{未来期望}

本文虽然在INRIA数据集上得到了较好的重建效果,基本满足人眼主观审美的标准。但在重建在算法性能与重建细节效果上仍有许多提升空间,后续工作会在以下几个方面尝试提升重建系统的性能和重建图像的主观效果:

\begin{enumerate}
\item 采用预处理的方式,云端每一幅图像的SIFT算子采用离线计算的方式预先进行提取,并使用压缩存储的方式进行存储,对常见的视觉词做缓存;
\item 离线训练是,采用Map-Reduce处理中间k-means环节,分布式并行加快学习速度;
\item 因为图像分享多用在旅游地标位置上,而互联网上本身在这些场景下的图像集较为充分,可以考虑GeoHash对包含GPS信息的图像集做搜索,在相似图像搜索环节,使用手机端的GPS信息进行更为快速的图像搜索。可以利用手机的时间、地理位置信息、当地天气等metadata做更为精细的索引操作;
\item 使用基于图的图像分割算法是迭代式的进行聚类操作,由于迭代算法不适用于多线程处理,因此图像分割是算法性能的瓶颈,尝试使用更小尺度更细化的图像块直接做完成筛选操作而不进行图像分割会带来速度上的提升了,但细化操作不当会引起错误的重建,如何选择细化方案是我们下一步的研究方向、
\end{enumerate}
