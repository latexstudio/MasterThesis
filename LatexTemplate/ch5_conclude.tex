%%
%% This is file `example/ch_concln.tex',
%% generated with the docstrip utility.
%%
%% The original source files were:
%%
%% install/buptgraduatethesis.dtx  (with options: `ch-concln')
%% 
%% This file is a part of the example of BUPTGraduateThesis.
%% 

\chapter{总结与展望}

\section{全文总结}
本文对当前的基于云的图像重建算法的各个环节进行了深入的探讨,在其基础上针对大规模图像数据集应用场景进行了一定的完善,在相似图像搜索和图像重建等算法中提出了一定的改进方案。

提出了自适应阈值的图像块筛选方法,实验结果表明:在不使用该方法时人工指定阈值在某些图像上能达到理想的还原效果,但在另外一些图像上则效果并不理想,结果数据显示由于不同图像之间不同区域的匹配块存在大量差异,没有一个固定的误差控制阈值能够适用于各类图像;而在重建系统中加入自适应阈值的方法,能够有效的找到合适的误差控制阈值,使得重建结果得到了提升,加强了图像重建系统的稳定性与适用性。

提出了图像分区和视觉词组编码进行相似图像搜索,实验结果显示分区后进行相似图像搜索显著提升了相关图像候选集的多样性,在大图像数据集上对图像重建效果有一定的提升。



\section{未来期望}

本文虽然在INRIA数据集上得到了较好的重建效果,基本满足人眼主观审美的标准。但在重建速度上仍有许多提升空间,后续工作会在以下几个方面尝试提升重建系统的性能和重建图像的主观效果:

\begin{enumerate}
\item 采用预处理的方式,云端每一幅图像的SIFT算子采用离线计算的方式预先进行提取,并使用压缩存储的方式进行存储,对常见的视觉词做缓存;
\item 离线训练是,采用Map-Reduce处理中间k-means环节,分布式并行加快学习速度;
\item 因为图像分享多用在旅游地标位置上,而互联网上本身在这些场景下的图像集较为充分,可以考虑GeoHash对包含GPS信息的图像集做搜索,在相似图像搜索环节,使用手机端的GPS信息进行更为快速的图像搜索。可以利用手机的时间、地理位置信息、当地天气等metadata做更为精细的索引操作;
\item 使用基于图的图像分割算法不方便多线程处理,尝试使用更小尺度更细化的图像块做图像处理;
\end{enumerate}