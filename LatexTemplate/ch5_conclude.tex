%%
%% This is file `example/ch_concln.tex',
%% generated with the docstrip utility.
%%
%% The original source files were:
%%
%% install/buptgraduatethesis.dtx  (with options: `ch-concln')
%% 
%% This file is a part of the example of BUPTGraduateThesis.
%% 

\chapter{结论与展望}

\section{实验结论}
本文对当前的基于云的图像重建算法的各个环节进行了深入的探讨,在其基础上针对大规模图像数据集应用场景进行了一定的完善,在相似图像搜索和图像重建等算法中提出了一定的改进方案。主要提出了自适应阈值的图像块筛选方法,实验结果表明:在不使用该方法时人工指定阈值在某些图像上能达到理想的还原效果,但在另外一些图像上则效果并不理想,结果数据显示由于不同图像之间不同区域的匹配块存在大量差异,没有一个固定的误差控制阈值能够适用于各类图像;而在重建系统中加入自适应阈值的方法,能够有效的找到合适的误差控制阈值,使得重建结果得到了提升,加强了图像重建系统的稳定性与适用性。

\section{并行优化}

可行方案:


* 预处理,每一幅图像的sift特征提取可以选择线下
   * 并行的可行性,Map-Reduce处理中间k-means环节

   * 分层索引加快在线查找效率:GeoHash


难点:


   * 使用基于图的图像分割算法不方便多线程处理
   * 怎样更充分的利用下采样信息做图像融合,找不到的区块怎样处理?是否可以考虑使用图像补全技术
