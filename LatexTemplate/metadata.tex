%%
%% This is file `example/metadata.tex',
%% generated with the docstrip utility.
%%
%% The original source files were:
%%
%% install/buptgraduatethesis.dtx  (with options: `metadata')
%% 
%% This file is a part of the example of BUPTGraduateThesis.
%% 

%% 涉密论文保密年限
\classdur{三年}

%% 学号
\studentid{2012110191}

%% 论文题目
\ctitle{基于局部特征的图像重建算法研究}
\etitle{Research on Local Features Based Image Reconstruction Algorithm}

%% 申请学位
\cdegree{工学硕士}

%% 院系名称
\cdepartment{信息与通信工程学院}

%% 专业名称
\cmajor{信号与信息处理}

%% 你的姓名
\cauthor{王继哲}

%% 你导师的姓名
\csupervisor{李学明}

%% 日期自动生成,也可以取消注释下面一行,自行指定日期
\cdate{\CJKdigits{2014}年\CJKnumber{12}月}

%% 中文摘要
\cabstract{%
基于局部特征的图像重建算法是利用原始图像的局部特征信息,以大图像集为数据源,进行较为精确的图像重建工作,使重建后的图像与原始图像相似,并且图像质量达到人眼主观效果较好的程度。图像重建算法在拷贝检测、隐私保护、超高分辨率重建、场景还原等领域有着广泛的应用。本文在基于云的图像压缩这一应用场景下对重建算法系统框架进行研究,对其中的技术细节进行了完善。

本文首先介绍基于云的图像压缩这一新颖的应用场景,着重介绍客户端特征提取与数据压缩、服务器端数据解码后使用大规模图像数据集进行图像重建这一完整的系统流程,介绍了系统中的各个模块及其对应的技术手段。其次从两个技术层面对目前图像重建系统加以阐述,包括(1)传统的全景图拼接技术中的图像局部特征、局部特征匹配、2D变换与特征配准、匹配特征块筛选以及图像融合等算法;(2)从大规模近似重复图像搜索角度叙述了视觉词袋模型、视觉词量化、最小哈希、视觉词组以及基于局部相关信息的相似图像搜索算法。

在完整的重建系统基础上,本课题针对重建系统在大规模语料集场景下的应用特点,对上述几个环节进行完善,主要包括在匹配图像块筛选中提出自适应阈值图像块筛选法、分区的相似图像查询、视觉词组的二维编码等,为重建提供了更有力的依据,构建了完整系统的仿真环境,从仿真结果来看重建效果达到了令人满意的程度。
}

%% 中文关键词,关键词之间用 \kwsep 分割
\ckeywords{图像重建 \kwsep 局部特征 \kwsep 图像分割 \kwsep 图像配准 \kwsep 图像融合 \kwsep 尺度不变特征变换 \kwsep 相似图像搜索}

%% 英文摘要
\eabstract{%
Local features based image reconstruction is an algorithm which uses local feature information of the original image,together with the large-scale image dataset,to perform accurate image reconstruction so that the reconstructed image is similar to the original image and achieves good visual quality.

The scenarios of cloud-base image coding was introduced firstly.We discuss the complete system process including feature extraction and data compression at the client end and reconstruct image by using decoding data and large-scale image dataset.Different modules and corresponding technologies are mentioned.

Secondly,we dive into the current architecture of the image reconstruction system from two technical level,including (1)the technologies of local descriptors,vfeature matching and registration,patch filtering, and image fusion,(2)isual word group,partial duplicate image retrieval.

Focusing on the features of the scenarios of large-scale corpus, we propose several technologies which include adaptive threshold validation,block-based similar query and 2-d visual words coding to optimize current system at the scene of large-scale corpus.The results demonstrate that the proposed methods provides stronger reconstruction evidence and thus improve the performance.
}

%% 英文关键词,也用 \kwsep 分割
\ekeywords{%
  Image reconstruction \kwsep local features \kwsep image segmentation \kwsep image registration \kwsep image fusion \kwsep sift \kwsep partial-duplicate image retrieval}
