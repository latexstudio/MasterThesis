%%
%% This is file `example/metadata.tex',
%% generated with the docstrip utility.
%%
%% The original source files were:
%%
%% install/buptgraduatethesis.dtx  (with options: `metadata')
%% 
%% This file is a part of the example of BUPTGraduateThesis.
%% 

%% 涉密论文保密年限
\classdur{三年}

%% 学号
\studentid{2012110191}

%% 论文题目
\ctitle{基于局部特征的图像重建算法研究}
\etitle{Research on image reconstruction algorithm based on local features}

%% 申请学位
\cdegree{工学硕士}

%% 院系名称
\cdepartment{信息与通信工程学院}

%% 专业名称
\cmajor{信号与信息处理}

%% 你的姓名
\cauthor{王继哲}

%% 你导师的姓名
\csupervisor{李学明}

%% 日期自动生成,也可以取消注释下面一行,自行指定日期
\cdate{\CJKdigits{2014}年\CJKnumber{12}月}

%% 中文摘要
\cabstract{%
  中、英文摘要位于声明的次页,摘要应简明表达学位论文的内容要点,体现研究工作的核心思想。
  重点说明本项科研的目的和意义、研究方法、研究成果、结论,注意突出具有创新性的成果和新见解的部分。

  关键词是为文献标引工作而从论文中选取出来的、用以表示全文主题内容信息的术语。
  关键词排列在摘要内容的左下方,具体关键词之间以均匀间隔分开排列,无需其它符号。
}

%% 中文关键词,关键词之间用 \kwsep 分割
\ckeywords{重建 \kwsep 局部特征 \kwsep 图像分割 \kwsep 图像配准 \kwsep 图像融合}

%% 英文摘要
\eabstract{%
  The Chinese and English abstract should appear after the declaration page.
  The abstract should present the core of the research work, especially the purpose and importance of the research, the method adopted, the results, and the conclusion.

  Key words are terms selected for documentation indexing, which should present the main contributions of the thesis.
  Key words are aligned at the bottom left side of the abstract content.
  Key words should be seperated by spaces but not any other symbols.
}

%% 英文关键词,也用 \kwsep 分割
\ekeywords{%
  \TeX \kwsep \LaTeX \kwsep xeCJK \kwsep template \kwsep typesetting \kwsep thesis}
