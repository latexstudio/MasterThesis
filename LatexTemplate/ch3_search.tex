%%
%% 第三章
%% 2012.5.22


\chapter{大规模相似图像搜索算法概述}

随着多媒体业务的日益增长,相似图像搜索(或称部分重复图像搜索Partial Duplicate Image Retrieval)技术得到了愈加广泛的应用。在我们的图像重建系统中的相似图像搜索环节,我们希望找到尽可能多的与用户拍摄图像相似的图像,将其作为后续重建环节的候选图像。因此我们面临的三个技术难点是:(1)相似搜索是在图像的局部进行的,而不是整幅图像,所以使用全局特征进行相似图像搜索的传统方案并不适用,是否有能表述局部特性的图像表示方法;(2)图像的局部特征信息较少,如何充分利用特征之间的几何位置关系进行图像局部匹配来提高搜索精度;(3)云端图像数据库是Web规模的(Web-Scale),图像数据量极大,对算法的时空复杂度限制较大。如何在使用图像局部特征和其空间位置关系的同时尽量不增加搜索算法的复杂度,是本系统需要解决的难题。

本章首先介绍传统的图像搜索算法,再介绍利用局部特征的空间信息的相似图像搜索算法,最后针对本论文的应用场景,提出一种结合多种技术的新的相似图像搜索技术。

\section{传统的大规模图像检索方法}

\subsection{相似图像搜索算法概述}

\subsection{图像表示与相似性度量}
%[相似度计算](http://blog.sina.com.cn/s/blog_6634c1410100w56x.html)

\subsection{基于局部特征的相似图像搜索算法}
BoW

聚类
tf-idf


\section{改进的相似搜索算法}
文献\cite{POLICY:2013te}对近期的大规模相似图像搜索技术做了总结,提到了Partial-Duplicate Image Retrieval via Saliency-Guided Visual Matching\cite{Li:2013ks}技术,
通过视觉显著性(saliency)模型进行比较,消除背景中的噪声。这种方法使得索引和匹配都集中在显著性区域,更能够符合用户的预期。显著值和空间约束都能够被用来进行相似性度量,并且能够高效的进行二级索引,对于大规模的partial duplicate search非常有利,但是内存开销比较大。

Web-Scale Image Retrieval Using Compact Tensor Aggregation of Visual Descriptors\cite{Negrel:2013ur}描述了目前存在的各种视觉描述子的概况,介绍了相关的索引技术,包括哈希、词袋以及基于树的表示方法。(hashing, bag-of-words, and tree-based representation)引出内存开销问题并提出一种生成高度压缩签名(highly compact signatures)的方法,包括张量聚合,PCA,kernel PCA等一些列算法。它改进了Fisher Vector族 描述子,提高它的可区分性,以及特征签名的大小(feature discrimina- tive power and the size of feature signature)。

对于相似性视频搜索,它的特点是特征维数特别大,有研究提出了稀疏投影方式进行特征降维,并且使用数据挖掘的知识使用一些metadata来共同进行搜索\cite{Wu:2013tb}。使用机器学习技术,学习稀疏投影矩阵(sparse projection matrices)。这种学习方法可以选择性的使用外部信息,比如WikiPedia上的知识和Google搜索结果中的摘要,创建一个语义相关的投影矩阵,生成一个压缩签名,以满足手机媒体检索的诸多限制。手机内存空间小,计算资源有限,传统的将高维特征映射到低维的投影矩阵在手机内存是放不下的。而我们的稀疏投影矩阵是能够在手机上使用的。


三、SimHash

%[海量数据相似度计算之simhash和海明距离](http://www.lanceyan.com/tech/arch/simhash_hamming_distance_similarity.html)

这篇文章介绍的简明扼要,这里直接将核内容粘贴过来

为此我们需要一种应对于海量数据场景的去重方案,经过研究发现有种叫 local sensitive hash 局部敏感哈希的东西,据说这玩意可以把文档降维到hash数字,数字两两计算运算量要小很多。查找很多文档后看到google对于网页去重使用的是simhash,他们每天需要处理的文档在亿级别,大大超过了我们现在文档的水平。既然老大哥也有类似的应用,我们也赶紧尝试下。simhash是由 Charikar 在2002年提出来的,参考[《Similarity estimation techniques from rounding algorithms》](dl.acm.org/citation.cfm?id=509965) 。 介绍下这个算法主要原理,为了便于理解尽量不使用数学公式,分为这几步:

1. 分词,把需要判断文本分词形成这个文章的特征单词。最后形成去掉噪音词的单词序列并为每个词加上权重,我们假设权重分为5个级别(1~5)。比如:“ 美国“51区”雇员称内部有9架飞碟,曾看见灰色外星人 ” ==> 分词后为 “ 美国(4) 51区(5) 雇员(3) 称(1) 内部(2) 有(1) 9架(3) 飞碟(5) 曾(1) 看见(3) 灰色(4) 外星人(5)”,括号里是代表单词在整个句子里重要程度,数字越大越重要。
2. **hash**,通过hash算法把每个词变成hash值,比如“美国”通过hash算法计算为 100101,“51区”通过hash算法计算为 101011。这样我们的字符串就变成了一串串数字,还记得文章开头说过的吗,要把文章变为数字计算才能提高相似度计算性能,现在是降维过程进行时。
3. **加权**,通过 2步骤的hash生成结果,需要按照单词的权重形成加权数字串,比如“美国”的hash值为“100101”,通过加权计算为“4 -4 -4 4 -4 4”;“51区”的hash值为“101011”,通过加权计算为 “ 5 -5 5 -5 5 5”。
4. 合并**,把上面各个单词算出来的序列值累加,变成只有一个序列串。比如 “美国”的 “4 -4 -4 4 -4 4”,“51区”的 “ 5 -5 5 -5 5 5”, 把每一位进行累加, “4+5 -4+-5 -4+5 4+-5 -4+5 4+5” ==》 “9 -9 1 -1 1 9”。这里作为示例只算了两个单词的,真实计算需要把所有单词的序列串累加。
5. **降维**,把4步算出来的 “9 -9 1 -1 1 9” 变成 0 1 串,形成我们最终的simhash签名。 如果每一位大于0 记为 1,小于0 记为 0。最后算出结果为:“1 0 1 0 1 1”。


得出的重要结论:

通过大量测试,simhash用于比较大文本,比如500字以上效果都还蛮好,距离小于3的基本都是相似,误判率也比较低。但是如果我们处理的是微博信息,最多也就140个字,使用simhash的效果并不那么理想。看如下图,在距离为3时是一个比较折中的点,在距离为10时效果已经很差了,不过我们测试短文本很多看起来相似的距离确实为10。如果使用距离为3,短文本大量重复信息不会被过滤,如果使用距离为10,长文本的错误率也非常高,如何解决?

总结:

- 按照Charikar在论文中阐述的,64位simhash,海明距离在3以内的文本都可以认为是近重复文本。当然,具体数值需要结合具体业务以及经验值来确定。 
- simhash还可以用于信息聚类、文件压缩
- 选择和设计文本的去重算法?常见的有余弦夹角算法、欧式距离、Jaccard相似度、最长公共子串、编辑距离等,但是只适合于小数据集
- simhash传统的用来判断两篇文章的相似度,将两篇文章映射到低维空间上,并且保持它们互相之间的相似度,但是它很难应用在图像比较上,因为图像的特征是用实数来表示的,尽管可以将其量化,但是两幅相似图像量化后的特征集合交叠的比率仍旧很小,远远小于文档,因为两幅图像不相似的区域的噪声特征非常大
- 但是如果使用bundle features,那么如果两个相似区域的bundle features会非常相同,我们就可以使用simhash了
- 所以,min-hash的使用场景是特征比较多,相似度比较显著的情况下。


使用min-hash得到摘要

- 思路转化:比较两幅图像或者两篇文章的相似度问题转化为比较两个只包含0,1元素的集合的相似度,集合的相似度是Jaccard相似度。\(JS(A,B)=\frac{|A\cap{B}|}{|A\cup{B}|}\)
- 根据这样一个神奇的公式\(Pr[m(S_i) = m(S_j)] = E[\hat{JS}] =JS(S_i,S_j)\),使用min-hash函数将一幅图片或者一篇文章转化为一个数(对该文章中的每一个单词id使用hash函数后得到一个新的id序列,这个序列中的第一个出现1的行号,就是min-hash的值)
- 这样我们可以使用k个hash函数,得到k个值,将原本的高维向量映射到了低维
- Min-hash在压缩原始集合的情况下,保证了集合的相似度没有被破坏。

使用LSH缩小查找范围

其基本思路是将相似的集合聚集到一起,减小查找范围,避免比较不相似的集合

对每一列c(即每个集合)我们都计算出了n行minhash值,我们把这n个值均分成b组,每组包含相邻的r=n/b行。对于每一列,把其每组的r个数都算一个hash值出来,把此列的编号记录到hash值对应的bucket里。如果两列被放到了同一个bucket里,说明它们至少有一组(r个)数的hash值相同,此时可认为它们有较大可能相似度较高(称为一对candidate)。最后在比较时只对落在同一个bucket里的集合两两计算,而不是全部的两两比较。

\section{空间信息匹配搜索算法}


\subsection{随机抽样一致}
随机抽样一致RANdom SAmple Consensus(RANSAC)是一种空间匹配算法。该算法将数据分成两类,局内点(inlier)和局外点(outlier)它可以从一组包含局外点的观测数据集中,通过迭代方式估计数学模型的参数。

这是一种不确定的算法,有一定的概率得出一个正确的或者说是可接受的合理结果;一般情况下,迭代次数的增加可以提升结果的准确性。该算法由Fischler和Bolles于1981年提出,在图像检索中,RANSAC可以作为检索后的后续处理,对图像的中目标进行空间一致验证。

RANSAC算法对数据集做了三个假设:

\begin{itemize}
\item 数据由局内点组成,局内点的数据的分布符合某一特定的概率模型;
\item 与局内点相对的是局外点,他们不能够适应该模型;
\item 局内点与局外点之外的数据属于噪声
\end{itemize}

RANSAC有以下几个步骤:
\begin{itemize}
\item 随机选择数据集合的一个子集
\item 使用选择的自己拟合一个数学模型
\item 确定该模型下局外点的个数
\item 重复步骤1~3若干次,以最好的一次结果最为最终拟合出来的数学模型
\end{itemize}

当对两幅图像进行匹配的时候,所以相互匹配的局部特征作为数据全集,我们要估算的模型是一个透视变换矩阵,能够将图像I投影到图像I`。每次迭代过程中,随机的选择四对匹配的特征点,根据这四个特征点的位置信息解得变换H,利用H计算其它匹配对的位置信息中有哪些属于局外点,记录局外点的个数。局外点的个数越少,变换矩阵H越准确。反复迭代多次得到一个相对准确的透视变换模型。

迭代次数的选取取决于我们期望的准确率与样本数量。设p为任意给定对应点合法的概率,即
\[p = \frac{\text{局内点的数量}}{\text{数据集全部数据的数量}}\]
而P是经过S次试验后成功的总体概率。设我们需要k个随机样本来估计模型,那么在一次试验中,该k个样本都是局内点的可能性为\(p^k\)。因此,S次试验失败的可能性是
\[1 - P = (1 - p^k)^S\]
两边去对数,得到最少需要的试验次数是
\[S = \frac{log(1-P)}{log(1-p^k)}\]
随着k的增大,需要的最少试验次数增多,在实际中,我们应该尽可能的选择小的k值。

\subsection{视觉词组}

BoW的一个被人诟病的缺点是没有利用任何的图像空间信息,而这正是图像搜索与文本搜索最显著的区别所在,目前,有很多研究成功的利用局部特征的空间位置信息进行更加精确的,文献\cite{Xu:2013wc}深入研究SIFT描述子。提出了一个非常优雅的方法:生成sift组,嵌入几何信息,最终讲一个group压缩到一个64比特的二维签名中,叫做Nested-SIFT。

它的优点是Nested—SIFT使用sift描述子的嵌套关系,很自然的将不同尺度的局部关键点组合在一起,生成一个特征签名。嵌入空间信息的Nested—SIFT可区分性更强。使用SimHash进行压缩后,在视觉搜索中效率更高。实验结果表明这种方法提高搜索的准确度,减少了内存消耗,提高搜索速度。其缺点是生成Nested-SIFT会有一定的计算消耗

\section{适用于旅游景点图像的相似图像搜索技术}

%% 本章参考文献
\ifx\usechapbib\empty
\nocite{BSTcontrol}
\bibliographystyle{buptgraduatethesis}
\bibliography{bare_thesis}
\fi