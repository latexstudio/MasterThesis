%%
%% 第三章
%% 2012.5.22


\chapter{大规模近似重复图像搜索算法概述}

随着多媒体业务的日益增长,近似重复图像搜索(Near Duplicate Image Retrieval)或部分重复图像搜索(Partial Duplicate Image Retrieval)技术得到了愈加广泛的应用。在我们的图像重建系统中的相似图像搜索环节,我们希望找到尽可能多的与用户拍摄图像相似的图像,将其作为后续重建环节的候选图像。因此我们面临的三个技术难点是:(1)相似搜索是在图像的局部进行的,而不是整幅图像,所以使用全局特征进行相似图像搜索的传统方案并不适用,是否有能表述局部特性的图像表示方法;(2)图像的局部特征信息较少,如何充分利用特征之间的几何位置关系进行图像局部匹配来提高搜索精度;(3)云端图像数据库是Web规模的(Web-Scale),图像数据量极大,对算法的时空复杂度限制较大。如何在使用图像局部特征和其空间位置关系的同时尽量不增加搜索算法的复杂度,是本系统需要解决的难题。

本章首先介绍传统的图像搜索算法,再介绍利用局部特征的空间信息的相似图像搜索算法,最后针对本论文的应用场景,提出一种结合多种技术的新的相似图像搜索技术。

\section{基于局部特征的相似图像搜索算法}

最常见的基于局部特征的相似搜索算法包含两个环节。第一步,从图像中提取局部特征,图像的感兴趣区域可以通过自动的特征点检测或者均匀取样获得,最常见的局部特征描述子包括梯度方向直方图(histograms of oriented gradient,HOG)和SIFT、SURF等。从一幅图像抽取的特征集合叫做视觉词袋(Bag of Visual Words)。在第二步中,我们需要定义两个视觉词袋之间的相似性,第一类是直接比较两个视觉词袋的相似性,例如投票方法;第二类是通过视觉词袋计算一个特征签名(signature,通常是一个向量),进而比较两个签名之间的相似度。两种方式都需要对数据库中的所有图像与请求的图像比较相似度并排序\cite{POLICY:2013te}。

\subsection{视觉词袋模型}
视觉词袋(Bag of Visual Words)模型是图像表示中最为经典的一种表示方法。它经常被用来进行图像分类和相似性搜索领域。它来自文档检索基于关键字查询的方法中词袋(Bag of Words)的表示方法,其基本思想是:(1)统计语料库中的所有单词,生成单词表;(2)对于每一篇文档,统计每一个单词出现的频次,用由这些单词出现的次数生成直方图,用直方图来表示这篇文档。这种直方图的表示就是词袋表示。视觉词袋类似于BoW模型,算法的基本思路如下:

(1)离线部分:
\begin{itemize}
\item 提取特征:根据使用场景与实际业务的不同,可以选择不同的特征,文献\cite{Zhang:2006ej}对视觉词袋模型进行深入的分析,综合比较了各种特征检测器、描述子等。在这一步,我们综合考虑特征的时空复杂度、鲁棒性、可区分性等。
\item 生成视觉词码表:统计图像数据库中出现的所有特征,去除冗余的特征(比如几乎每一篇文档中都会出现的特征,类似于文档中的停用词)组成视觉词码表(Codebook)。如果提取的图像特征过多,一般需要对特征进行量化,利用聚类算法先把相近的单词归为一类(类似于文档检索里的找词根),利用聚类的结果生成视觉词码表。
\item 利用视觉词码表量化所有的图像特征
\item 利用词频表示的词袋模型来表示数据库中的每一幅图像
\end{itemize}

(2)在线部分:
\begin{itemize}
\item 提取请求图像的局部特征;
\item 利用视觉词码表量化该图像的图像特征
\item 利用词频表示的词袋模型来表示请求图像
\item 利用词频表示做进一步的处理,例如分类,相似性比较等
\end{itemize}

\includegraphics[width=5.00cm]{imgs/ch3/histogram}

\subsection{局部特征的量化}

\subsection{相似性度量}
%[相似度计算](http://blog.sina.com.cn/s/blog_6634c1410100w56x.html)




\section{改进的相似搜索算法}
文献\cite{POLICY:2013te}对近期的大规模相似图像搜索技术做了总结,提到了Partial-Duplicate Image Retrieval via Saliency-Guided Visual Matching\cite{Li:2013ks}技术,
通过视觉显著性(saliency)模型进行比较,消除背景中的噪声。这种方法使得索引和匹配都集中在显著性区域,更能够符合用户的预期。显著值和空间约束都能够被用来进行相似性度量,并且能够高效的进行二级索引,对于大规模的partial duplicate search非常有利,但是内存开销比较大。

Web-Scale Image Retrieval Using Compact Tensor Aggregation of Visual Descriptors\cite{Negrel:2013ur}描述了目前存在的各种视觉描述子的概况,介绍了相关的索引技术,包括哈希、词袋以及基于树的表示方法。(hashing, bag-of-words, and tree-based representation)引出内存开销问题并提出一种生成高度压缩签名(highly compact signatures)的方法,包括张量聚合,PCA,kernel PCA等一些列算法。它改进了Fisher Vector族 描述子,提高它的可区分性,以及特征签名的大小(feature discrimina- tive power and the size of feature signature)。

对于相似性视频搜索,它的特点是特征维数特别大,有研究提出了稀疏投影方式进行特征降维,并且使用数据挖掘的知识使用一些metadata来共同进行搜索\cite{Wu:2013tb}。使用机器学习技术,学习稀疏投影矩阵(sparse projection matrices)。这种学习方法可以选择性的使用外部信息,比如WikiPedia上的知识和Google搜索结果中的摘要,创建一个语义相关的投影矩阵,生成一个压缩签名,以满足手机媒体检索的诸多限制。手机内存空间小,计算资源有限,传统的将高维特征映射到低维的投影矩阵在手机内存是放不下的。而我们的稀疏投影矩阵是能够在手机上使用的。

下面我们简单介绍各种性能改进的方法。

\subsection{sim-hash}
文本的去重算法中常见的有余弦夹角算法、欧式距离、Jaccard相似度、最长公共子串、编辑距离等,但是只适合于小数据集。simhash传统的用来判断两篇文章的相似度,将两篇文章映射到低维空间上,并且保持它们互相之间的相似度,但是它很难应用在图像比较上,因为图像的特征是用实数来表示的,尽管可以将其量化,但是两幅相似图像量化后的特征集合交叠的比率仍旧很小,远远小于文档,因为两幅图像不相似的区域的噪声特征非常大。但是如果使用视觉词组,那么如果两个相似区域的视觉词组会非常相同,我们就可以使用simhash了。所以,min-hash的使用场景是特征比较多,相似度比较显著的情况下。

\subsection{最小哈希的相似性比较}
最小哈希方法是一种广泛应用在相似性查找领域的算法。

在生成视觉词带之后,比较两幅图像或者两篇文章的相似度问题转化为比较两个只包含0,1元素的集合的相似度,集合的相似度是Jaccard相似度。
\[JS(A,B)=\frac{|A\cap{B}|}{|A\cup{B}|}\]
我们首先定义一组随机哈希函数\(f_j:\mathcal{V} \to R\),每一个哈希函数是独立的,将一个视觉词映射为一个实数。两个不同的视觉词\(X_a\)和\(X_b\),哈希函数需要满足两点:(1)\(f_j(X_a)\neq f_j(X_b)\);(2)\(P(f_j(X_a)) < (P(f_j(X_b)) = 0.5\)。

注意到函数\(f_j\)能够反映出视觉词集合的一个顺序排列情况,我们定义min-hash就是这个排列中排在最前面的视觉词,因此有
\[m(\mathcal{A}_i,f_j)= arg\mathop {\min }\limits_{X \in A_i}f_j(X)\]
根据上述定义,我们会发现以下这个事实:
\[Pr(m(A,f_j) = (B,f_j)) = \frac{|A\cap{B}|}{|A\cup{B}|} = sim_s(A,B)\]

如果r是随机变量,当\(m(A,f_j) = (B,f_j)\)时值为1,其它情况下值为0的,那么r可认为是J(A,B)的无偏估计。因此我们可以使用min-hash函数将一幅图片或者一篇文章转化为一个数(对该文章中的每一个单词id使用hash函数后得到一个新的id序列,这个序列中的第一个出现1的行号,就是min-hash的值),当我们使用k个hash函数,得到k个值,将原本的高维向量映射到了低维。min-hash在压缩原始集合的情况下,保证了集合的相似度没有被破坏。
文献\cite{Chum:2008jo}中提出将词频-逆文档频率(term frequency–inverse document frequency,简称TF-IDF)融合在传统的min-hash算法中,实验表明能够提高搜索的准确率。

\subsection{LSH}
使用min-hash对数据降维后,可以使用LSH缩小查找范围,其基本思路是将相似的集合聚集到一起,减小查找范围,避免比较不相似的集合。

对每一列c(即每个集合)我们都计算出了n行min-hash值,我们把这n个值均分成b组,每组包含相邻的r=n/b行。对于每一列,把其每组的r个数都算一个hash值出来,把此列的编号记录到hash值对应的bucket里。如果两列被放到了同一个bucket里,说明它们至少有一组(r个)数的hash值相同,此时可认为它们有较大可能相似度较高(称为一对candidate)。最后在比较时只对落在同一个bucket里的集合两两计算,而不是全部的两两比较。

\section{空间信息匹配搜索算法}
使用视觉词袋模型来表示图像并比较视觉词袋之间的相似性做法比较成熟、最为普及的做法
\subsection{随机抽样一致算法}
随机抽样一致RANdom SAmple Consensus(RANSAC)是一种空间匹配算法。该算法将数据分成两类,局内点(inlier)和局外点(outlier)它可以从一组包含局外点的观测数据集中,通过迭代方式估计数学模型的参数。

这是一种不确定的算法,有一定的概率得出一个正确的或者说是可接受的合理结果;一般情况下,迭代次数的增加可以提升结果的准确性。该算法由Fischler和Bolles于1981年提出,在图像检索中,RANSAC可以作为检索后的后续处理,对图像的中目标进行空间一致验证。

RANSAC算法对数据集做了三个假设:

\begin{itemize}
\item 数据由局内点组成,局内点的数据的分布符合某一特定的概率模型;
\item 与局内点相对的是局外点,他们不能够适应该模型;
\item 局内点与局外点之外的数据属于噪声
\end{itemize}

RANSAC有以下几个步骤:
\begin{itemize}
\item 随机选择数据集合的一个子集
\item 使用选择的自己拟合一个数学模型
\item 确定该模型下局外点的个数
\item 重复步骤1~3若干次,以最好的一次结果最为最终拟合出来的数学模型
\end{itemize}

RANSAC算法迭代次数的选取取决于我们期望的准确率与样本数量。设p为任意给定对应点合法的概率,即
\[p = \frac{\text{局内点的数量}}{\text{数据集全部数据的数量}}\]
而P是经过S次试验后成功的总体概率。设我们需要k个随机样本来估计模型,那么在一次试验中,该k个样本都是局内点的可能性为\(p^k\)。因此,S次试验失败的可能性是
\[1 - P = (1 - p^k)^S\]
两边去对数,得到最少需要的试验次数是
\[S = \frac{log(1-P)}{log(1-p^k)}\]
随着k的增大,需要的最少试验次数增多,在实际中,我们应该尽可能的选择小的k值。在模型确定以及最大迭代次数允许的情况下,RANSAC总是能找到最优解。对于含有较大误差的数据集,RANSAC的效果远优于直接的最小二乘法。 

当对两幅图像进行匹配的时候,所以相互匹配的局部特征作为数据全集,我们要估算的模型是一个变换矩阵H,能够将图像\(I\)投影到图像\(I'\)。每次迭代过程中,随机的选择四对匹配的特征点,根据这四个特征点的位置信息解得变换H,利用H计算其它匹配对的位置信息中有哪些属于局外点,记录局外点的个数。局外点的个数越少,变换矩阵H越准确。反复迭代多次得到一个相对准确的透视变换模型。

上述提到的用四对匹配点拟合出的变换矩阵叫做单应矩阵(Homography),最简单的求解单应性矩阵的算法叫做直接线性变换法(Direct Linear Transform,DLT)\cite{Dubrofsky:2009tz},其具体算法如下:

假设我们相匹配的一对点分别是\(x\)和\(x'\),单应性矩阵是\(H\),那么有如下等式:

\begin{equation}
\label{homography1}
	c
	\begin{pmatrix}
	u \\
	v \\
	1
	\end{pmatrix}
	= H
	\begin{pmatrix}
	x \\
	y \\
	1
	\end{pmatrix}
\end{equation}


其中c是一个非零常数,\((u\ v\ 1)^\mathrm{T}\)代表\(x'\),\((x \ y \ 1)^\mathrm{T}\)代表\(x\),而
\(
H = 
\begin{pmatrix}
h_1 & h_2 & h_3 \\
h_4 & h_5 & h_6 \\
h_7 & h_8 & h_9
\end{pmatrix}
\)

将公式\eqref{homography1}展开,分别用第一行和第二行除以第三行,得到

\begin{equation}
\label{homography2}
-h_1x - h_2y - h_3 + (h_7x+h_8y+h_9)u = 0
\end{equation}

\begin{equation}
\label{homography3}
-h_4x - h_5y - h_6 + (h_7x+h_8y+h_9)u = 0
\end{equation}

公式\eqref{homography2}和\eqref{homography3}可以写成矩阵的形式:
\begin{equation}
\label{homography4}
A_ih = 0
\end{equation}
其中
\(
A_i = 
\begin{pmatrix}
-x & -y & -1 & 0 & 0 & 0 &ux & uy & u \\
0 & 0 & 0 & -x & -y & -1 &vx & vy & v 
\end{pmatrix}
\)
,而
\(
h = (h_1 \ \ h_2 \ \ h_3 \ \ h_4 \ \ h_5 \ \ h_6 \ \ h_7 \ \ h_8 \ \ h_9)^\mathrm{T}
\)。
因为每一对匹配的点可以提供两个等式,对于解决8自由度的矩阵H,只需要四对(任意三点不能共线)匹配的特征点。

DLT算法依赖于坐标系的原点和尺度,所以该算法并不稳定,在实际中更多的使用多个匹配点得到更多的方程,将求单应矩阵的问题转化为求解最小二乘的问题,用矩阵奇异值分解(Singular value decomposition,SVD)的方法来求解等式\(Ah = 0\)。

\subsection{视觉词组}

BoW的一个被人诟病的缺点是没有利用任何的图像空间信息,而这正是图像搜索与文本搜索最显著的区别所在,目前,有很多研究成功的利用局部特征的空间位置信息进行更加精确的,文献\cite{Xu:2013wc}深入研究SIFT描述子。提出了一个非常优雅的方法:生成sift组,嵌入几何信息,最终讲一个group压缩到一个64比特的二维签名中,叫做Nested-SIFT。

它的优点是Nested—SIFT使用sift描述子的嵌套关系,很自然的将不同尺度的局部关键点组合在一起,生成一个特征签名。嵌入空间信息的Nested—SIFT可区分性更强。使用SimHash进行压缩后,在视觉搜索中效率更高。实验结果表明这种方法提高搜索的准确度,减少了内存消耗,提高搜索速度。其缺点是生成Nested-SIFT会有一定的计算消耗。

\section{适用于旅游景点图像的相似图像搜索技术}

%% 本章参考文献
\ifx\usechapbib\empty
\nocite{BSTcontrol}
\bibliographystyle{buptgraduatethesis}
\bibliography{bare_thesis}
\fi