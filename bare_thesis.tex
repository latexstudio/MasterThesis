%%
%% This is file `example/bare_thesis.tex',
%% generated with the docstrip utility.
%%
%% The original source files were:
%%
%% install/buptgraduatethesis.dtx  (with options: `bare-thesis')
%% 
%% This file is a part of the example of BUPTGraduateThesis.
%% 

\documentclass[%
  degree=master,%
  classlevel=open,%
  mathfont=mathptmx,%
  dedication=false,%
  chapbib=true,%
  finish=print,%
  driver=xetex]{buptgraduatethesis}

%% 自定义导言区
%% 在这里添加你需要的宏包、自定义命令、环境等
%% \usepackage{...}
%% \DeclareMathOperator{\CT}{H}
%% \DeclareMathOperator{\Cov}{Cov}
\def\BUPTThesis{\textsc{BUPT}\-\textsc{Thesis}}

%% 在这里添加图片文件搜索目录
\graphicspath{{../}}
%% 自定义导言区结束

%% 加载缩略语定义
%%
%% This is file `example/metadata.tex',
%% generated with the docstrip utility.
%%
%% The original source files were:
%%
%% install/buptgraduatethesis.dtx  (with options: `metadata')
%% 
%% This file is a part of the example of BUPTGraduateThesis.
%% 

%% 涉密论文保密年限
\classdur{三年}

%% 学号
\studentid{2012110191}

%% 论文题目
\ctitle{基于局部特征的图像重建算法研究}
\etitle{Research on image reconstruction algorithm based on local features}

%% 申请学位
\cdegree{工学硕士}

%% 院系名称
\cdepartment{信息与通信工程学院}

%% 专业名称
\cmajor{信号与信息处理}

%% 你的姓名
\cauthor{王继哲}

%% 你导师的姓名
\csupervisor{李学明}

%% 日期自动生成,也可以取消注释下面一行,自行指定日期
\cdate{\CJKdigits{2014}年\CJKnumber{12}月}

%% 中文摘要
\cabstract{%
  中、英文摘要位于声明的次页,摘要应简明表达学位论文的内容要点,体现研究工作的核心思想。
  重点说明本项科研的目的和意义、研究方法、研究成果、结论,注意突出具有创新性的成果和新见解的部分。

  关键词是为文献标引工作而从论文中选取出来的、用以表示全文主题内容信息的术语。
  关键词排列在摘要内容的左下方,具体关键词之间以均匀间隔分开排列,无需其它符号。
}

%% 中文关键词,关键词之间用 \kwsep 分割
\ckeywords{重建 \kwsep 局部特征 \kwsep 图像分割 \kwsep 图像配准 \kwsep 图像融合}

%% 英文摘要
\eabstract{%
  The Chinese and English abstract should appear after the declaration page.
  The abstract should present the core of the research work, especially the purpose and importance of the research, the method adopted, the results, and the conclusion.

  Key words are terms selected for documentation indexing, which should present the main contributions of the thesis.
  Key words are aligned at the bottom left side of the abstract content.
  Key words should be seperated by spaces but not any other symbols.
}

%% 英文关键词,也用 \kwsep 分割
\ekeywords{%
  \TeX \kwsep \LaTeX \kwsep xeCJK \kwsep template \kwsep typesetting \kwsep thesis}


\loadglsentries{acronyms}

%% 攻读学位期间发表论文
%% 用 \newcite{<suffix>}{<caption>} 声明不同的论文类型(例如: 期刊论文、会议论文等)。每一个类型的对应的 .bib 文件用 \bibliography<suffix> 命令加载,用 \nocite<suffix> 命令引用。具体请参考 pubs.tex 中的示例
\newcite{jrnl}{期刊论文}
\newcite{conf}{会议论文}

\begin{document}
%% 声明前置部分
\makefrontmatter

%% 生成主要符号对照表
%%
%% This is file `example/notations.tex',
%% generated with the docstrip utility.
%%
%% The original source files were:
%%
%% install/buptgraduatethesis.dtx  (with options: `notations')
%% 
%% This file is a part of the example of BUPTGraduateThesis.
%% 

\begin{listofnotations}
\item [$(\cdot)^*$] 复共轭
\item [$(\cdot)^{\mathrm T}$] 矩阵转置
\item [$(\cdot)^{\mathrm H}$] 矩阵共轭转置
\item [$\mathbf{X}$] 矩阵或向量
\item [$\mathcal{A}$] 集合
\item [$\mathcal{A}\times\mathcal{B}$]
  集合 $\mathcal{A}$ 与集合 $\mathcal{B}$ 的 Cartesian 积,即 $\mathcal{A}\times\mathcal{B}=\{(a,b):a\in\mathcal{A},b\in\mathcal{B}\}$
\end{listofnotations}


%% 主体部分
\mainmatter
%% 用\include{}命令引用各章.tex文件
%%
%% This is file `example/ch_intro.tex',
%% generated with the docstrip utility.
%%
%% The original source files were:
%%
%% install/buptgraduatethesis.dtx  (with options: `ch-intro')
%% 
%% This file is a part of the example of BUPTGraduateThesis.
%% 

\chapter{绪论}
北京邮电大学 \gls*{BUPT} 研究生院培养与学位办公室于 2010 年 3 月 1 日颁布了最新的《北京邮电大学关于研究生学位论文格式的统一要求》(下简称“要求”)\cite{BUPT_Thesis_Format_2010},对原有研究生学位论文的格式要求做出了新的修订。
但是迄今为止,研究生院尚未发布统一的论文模板。
对于已经、正在或者即将撰写学位论文的同学都只能按照该要求的规定自行调整其学位论文的格式,一方面给大家增加了繁重的排版工作,另一方面也不利于统一全校的论文格式。

2007 年 9 月,北京邮电大学无线新技术研究所 \gls*{WTI} 的王旭博士制作并发布了 latex-bupt——北京邮电大学博士毕业论文 \LaTeX 模板(非官方版)\cite{latex-bupt}。
该模板可以满足旧版官方论文格式要求\cite{BUPT_Thesis_Format_2004},但是在一些细节上的处理还有待改进,例如:
\begin{itemize}
\item 参考文献不能分列在各章末尾;
\item 不能利用 BiBTeX 处理发表学术论文列表;
\item 参考文献的格式上赏不能完全满足学校要求等。
\end{itemize}

2009 年,张煜博士发布了 buptthesis——北京邮电大学研究生学位论文 \LaTeX 文档类(非官方版)\cite{buptthesis}。
该模板解决了 latex-bupt 中存在的问题,并且同样可以满足旧版官方论文格式要求\cite{BUPT_Thesis_Format_2004},但是仍然存在以下一些问题可以改进:
\begin{itemize}
\item 论文格式与最新版的官方论文格式要求\cite{BUPT_Thesis_Format_2010}有细微出入;
\item 中文解决方案采用旧式 CJK 宏包,需要用户自行生成字体;
\item 缺乏详细的用户使用文档,用户撰写论文过程中遇到的问题基本都需要登陆北邮人论坛发问,由张博逐一解答。
\end{itemize}

本模板在 buptthesis\cite{buptthesis} 的基础上,增加了 XeTeX 编译引擎,使用 xeCJK 宏包作为中文解决方案。
同时,本模板还根据北京邮电大学发布的最新的论文格式要求\onlinecite{BUPT_Thesis_Format_2010}进行模板格式的修改。
本模板还提供了较为细致的用户使用文档,可以帮助初级用户快速上手使用本模板。

\section{中文信息处理软件的国内外发展现状}
中文信息处理软件可以分为字处理软件和排版软件两大类。
字处理软件包括以下功能:字体、字号设定,英文断字,拼写和语法检查等。
通常字处理软件处理文档的规模比较小,一般是作为办公自动化套件的一个重要组成部分,目前广泛使用的中文字处理软件主要包括微软 Office 套件中的 Word、金山公司的 WPS,以及开源社区的 OpenOffice 等。
排版软件则是针对大规模专业出版印刷而设计的一类软件,其主要功能是文字图像定位,基本图形绘制等。
排版软件相对于字处理软件其专业针对性更强,目前广泛使用的中文排版软件主要包括北大方正的书版系列软件,飞腾系列软件,蒙泰桌面出版系统,Adobe 公司的 PageMaker,FrameMaker,以及 QuarkPress 公司的 PassPort 等。
除此而外,由 D.~E.~Knuth 编写的 \TeX 和由 L.~Lamport 编写的 \LaTeX 也是学术界广泛的应用排版软件。

微软公司的 Word 是目前国内最为普及的字处理软件之一,也是大多数学校规定的学位论文编辑排版工具。
不容否认,Word 在简单文书(例如:通知、简报等)编辑排版方面具有方便快捷的优势,而且其对多人协同编辑的支持也给文字修订工作带来了极佳的用户体验。
但是从实际使用的情况看,尽管 Word 已经经历了第 12 个版本的改进,但是其对于处理大型文书文稿(例如:书籍、学位论文等)的能力仍然有待进一步完善和提高。
由于 Word 版本不兼容造成的来回反复,也是使用 Word 编辑文字稿件的烦事之一。
另外,由于 Word 对数学公式编辑的支持一直延续其“对象链接与嵌入”(Object Linking and Embedding,OLE)的设计理念,这也使得每位使用 Word 排版过理工类的文字资料的人都有一段或多段刻骨铭心的痛苦经历,往往花在调整格式这种 dirty work 上的时间和花在编写文章内容上的时间差不多或着甚至更多。

北大方正的书版系列软件是专业中文出版领域的权威,国内几乎所有的大型出版社、报社、政府机关几乎都使用书版系列软件对其出版的书籍、报纸和公文进行编辑排版。
但是,书版软件作为方正电子出版流程中的一个主要组成部分,主要定位于印前排版环节,面向专业排版工作人员。
因此,学习和使用使用书版软件需要花费较长的时间来熟悉复杂的排版命令,发排后需要使用专用的 RIP 软件或者方正的专用打印机才能输出样张等。

美国 Stanford 大学的荣誉退休教授 D.~E.~Knuth 在 197x 年独自一人开发了 \TeX 排版系统,随后,L.~Lamport 为 \TeX 编写了一系列的宏包使得 \TeX 的使用更加方便,这些宏包被称为 \LaTeX。
自从 \TeX/\LaTeX 问世以来它们就受到了学术界的青睐,目前几乎所有的国外出版社都接受或指定使用 \TeX/\LaTeX 对稿件进行排版编辑。
19xx 年,中国科学院的张林波研究员开发了 CCT 使得 \LaTeX 可以用于中文文稿的处理。
德国的 W.~Lemberg,编写了 CJK 宏包为 \LaTeX 提供了中日韩三国语言的解决方案。
使用 \TeX/\LaTeX 排版学术论文的最大优势在于,它让作者可以不用为排版输出的具体格式操心,而全心投入文章、书稿内容的编写上,最大程度的降低作者从事排版 dirty work 的工作量。

目前,我国的清华大学、哈尔滨工业大学、西安电子科技大学、西安交通大学等都已经纷纷制作了本校学位论文的 \LaTeX 模板,并接受使用 \LaTeX 排版的学位论文。

\section{本说明的主要内容}
本说明全面介绍了如何使用 BUPTGraduateThesis 来排版符合\onlinecite{BUPT_Thesis_Format_2010}规定的北京邮电大学学位论文。
全文内容安排如下:

\begin{enumerate}
\item 第二章介绍……
\item ……
\end{enumerate}

%% 本章参考文献
\ifx\usechapbib\empty
\nocite{BSTcontrol}
\bibliographystyle{buptgraduatethesis}
\bibliography{bare_thesis}
\fi

%%
%% This is file `example/ch_concln.tex',
%% generated with the docstrip utility.
%%
%% The original source files were:
%%
%% install/buptgraduatethesis.dtx  (with options: `ch-concln')
%% 
%% This file is a part of the example of BUPTGraduateThesis.
%% 

\chapter{功能测试}
脚注使用带圈数字的表示方法,此处为示例 1\footnote{测试脚注一} 和示例 2\footnote{测试脚注二}。

缩略语的功能非常强大,例如首次出现 \gls*{WTT} 和非首次出现 \gls*{WTT} 时将显示不同的内容。

参考文献可以使用\cite{Dai:2012vn}和\onlinecite{BUPT_Thesis_Format_2004}的表示方法。

\section{三国演义}
《三国演义》\cite{Li:2008vn}是中国第一部长篇章回体历史演义的小说,以描写战争为主,反映了蜀(汉)、魏、吴三个政治集团之间的政治和军事斗争,大致分为黄巾之乱、董卓之乱、群雄逐鹿、三国鼎立、三国归晋五大部分。

在广阔的背景下,上演了一幕幕波澜起伏、气势磅礴的战争场面,成功刻画了近五百个人物形象,其中曹操、刘备、孙权、诸葛亮、周瑜、关羽、张飞等人物形象脍炙人口,其中诸葛亮是作者心目中的“贤相”的化身,他具有“鞠躬尽瘁,死而后已”的高风亮节,具有近世济民再造太平盛世的雄心壮志,而且作者还赋予他呼风唤雨、神机妙算的奇异本领。
曹操是一位奸雄,他生活的信条是“宁教我负天下人,休教天下人负我”,既有雄才大略,又残暴奸诈,是一个政治野心家阴谋家这与历史上的真曹操是不可混同的。
关羽“威猛刚毅”、“义重如山”。
但他的义气是以个人恩怨为前提的,并非国家民族之大义。
刘备被作者塑造成为仁民爱物、视贤下士、知人善任的仁君典型。

\subsection{长坂坡}
京剧《长坂坡》\cite{Zhao:2013ju}是依据《三国演义》改编的京剧传统剧目。

故事叙述:刘备自烧屯新野之后,弃樊城,阻襄阳,一路率引军民,流离败走,穷促万分。
关羽、诸葛亮,已先后遣往夏口,乞救于刘琦未返,刘备等往投江陵暂驻,中途经过当阳,驻扎景山之下。
忽然曹操大兵,漫山遍野追至,夤夜厮杀,刘备众大败,及天明检点随从只余百余骑,刘备家眷及赵云、简雍、二糜等将,均不知下落,其余百姓,亦均散失殆尽。
此时赵云因于阿斗及甘、糜二夫人等失散,遂单骑冲突,四处找寻主眷,沓无下落。
往回三数次,遇见简雍被创卧地,始略知失踪处所。
赵云先救出简雍,令回,再往军中及百姓中搜访,先救甘夫人于难民队,同时又救糜竺,亲自护送至长坂坡,令糜竺保甘嫂先行,折身再回,觅糜嫂及阿斗。
途中刺落夏侯恩,收获青釭宝剑,七次冲入重围,方得百姓指引,得见糜夫人抱阿斗坐于坍墙枯井之旁啼哭。
夫人身受数创,不能行走。
赵云叩见,极力请夫人上马,欲保护而出。
夫人深知大义,惟以阿斗为托,己则以愿死报主,免累赵云,赵云再三安慰催行,力任无妨,夫人再三不可,亦促赵云速行。
继见赵云坚待不去,恐且迟延遇寇,乃跳身入井,以速赵云之行。
赵云大惊,尚踌躇设法营救,则曹军人马已至,不得已推墙掩井,解甲藏阿斗于胸前,忽忽上马,厮杀夺围欲出。
此时曹操大兵云集,群矢于赵云一身,赵云在核心,东斩西杀,虽不败辱,而屡濒于厄。
幸曹操爱勇将,赖徐庶乘间说曹操,以生擒勿伤,传令全军,始得完肤而返。

%% 本章参考文献
\ifx\usechapbib\empty
\nocite{BSTcontrol}
\bibliographystyle{buptgraduatethesis}
\bibliography{bare_thesis}
\fi


%% 附录部分

%% 如果有两个或两个以上的附录, 使用appendix环境
\begin{appendix}
  %%
%% This is file `example/app_lhospital.tex',
%% generated with the docstrip utility.
%%
%% The original source files were:
%%
%% install/buptgraduatethesis.dtx  (with options: `app-lhospital')
%% 
%% This file is a part of the example of BUPTGraduateThesis.
%% 

\chapter{不定型($0/0$)极限的计算}
\begin{theorem}[L'Hospital法则]
  若
  \begin{enumerate}
  \item 当 $x \to a$ 时,函数 $f(x)$ 和 $g(x)$ 都趋于零;
  \item 在点 $a$ 某去心邻域内,$f'(x)$ 和 $g'(x)$ 都存在,且 $g'(x)\neq 0$;
  \item $\displaystyle\lim_{x \to a} \dfrac{f'(x)}{g'(x)}$ 存在(或为无穷大),
  \end{enumerate}
  那么
  \begin{align}
    \label{eq:app:lhospital}
    \lim_{x \to a} \frac{f(x)}{g(x)} = \lim_{x \to a} \frac{f'(x)}{g'(x)}.
  \end{align}
\end{theorem}
\begin{proof}
  以下只证明两函数 $f(x)$ 和 $g(x)$ 在 $x = a$ 为光滑函数的情形。
  由于 $f(a) = g(a) = 0$,原极限可以重写为
  \begin{align*}
    \lim_{x \to a} \frac{f(x) - f(a)}{g(x) - g(a)}.
  \end{align*}
  对分子分母同时除以 $(x - a)$,得到
  \begin{align*}
    \lim_{x \to a} \frac{%
      \dfrac{f(x) - f(a)}{x - a}
    }{%
      \dfrac{g(x) - g(a)}{x - a}
    } &
    = \frac{%
      \displaystyle\lim_{x \to a} \frac{f(x) - f(a)}{x - a}
    }{%
      \displaystyle\lim_{x \to a} \frac{g(x) - g(a)}{x - a}
    }.
  \end{align*}
  分子分母各得一差商极限,即函数 $f(x)$ 和 $g(x)$ 分别在 $x = a$ 处的导数
  \begin{align*}
    \lim_{x \to a} \frac{f(x)}{g(x)} &
    = \frac{f'(a)}{g'(a)}.
  \end{align*}
  由光滑函数的导函数必为一光滑函数,故 \eqref{eq:app:lhospital} 得证。
\end{proof}

  % 自动抽取生成缩略语表作为附录A
  \tableofacronyms
  % 用\input{}添加其他的附录
  % \input{...}
\end{appendix}

%% 如果只有一个附录, 使用appendix*环境
%% \begin{appendix*}
%%   % 自动抽取生成缩略语表作为附录A
%%   % \tableofacronyms
%% \end{appendix*}

\ifx\usechapbib\undefined
\bibliographystyle{buptgraduatethesis}
\bibliography{bare_thesis}
\fi

\backmatter
%% 致谢
\ifx\ispeerreview\undefined
%%
%% This is file `example/ackgmt.tex',
%% generated with the docstrip utility.
%%
%% The original source files were:
%%
%% install/buptgraduatethesis.dtx  (with options: `ackgmt')
%% 
%% This file is a part of the example of BUPTGraduateThesis.
%% 

\begin{acknowledgement}
  %% 感谢所有你应该感谢的人
  感谢Donald Ervin Knuth.
\end{acknowledgement}

\fi

%% 在读期间论文发表情况
%%
%% This is file `example/pubs.tex',
%% generated with the docstrip utility.
%%
%% The original source files were:
%%
%% install/buptgraduatethesis.dtx  (with options: `pubs')
%% 
%% This file is a part of the example of BUPTGraduateThesis.
%% 

%% 发表论文列表

%% 攻读学位期间发表论文列表用 tableofpublications 环境产生。需要
%% 在 bare_thesis.tex 的导言区用 \newcite{<name>}{<caption>} 声明不同类
%% 型的论文,具体见导言区说明。
\begin{tableofpublications}
  \thispagestyle{bupt@pubheadings}%
  \bibliographyjrnl{pubs}
  \nocitejrnl{paper1}

  \bibliographyconf{pubs}
  \nociteconf{paper2}
\end{tableofpublications}


\newpage
\end{document}
